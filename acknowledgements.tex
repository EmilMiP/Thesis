The work presented in this dissertation was carried out during my employment at the National Centre for Register-based Research (NCRR), Aarhus BSS, Aarhus University, while being enrolled at the Aarhus University Graduate School of Health. The work was funded in part by the Niels Bohr professorship to John McGrath and in part by the Graduate School of Health.

First, I would like to thank my supervisors. Bjarni Jóhann Vilhjálmsson for introducing me to the field of statistical genetics and agreeing to be my main supervisor for this PhD. Florian Franck Privé for sharing his extensive knowledge about R and statistics and helping me become a far better programmer. Esben Agerbo for his insight and encouragements. I would also like to thank all my supervisors for the many discussions and always being able to find time for me. In particular, I appreciate the support I received during the difficult times with lockdowns during the Covid pandemic. 

I would also like to thank Mark Daly for inviting me to come work with FinnGen data for a time. I found it very interesting and I hope to do similar things in the future. Thank you to Andrea Ganna for hosting me during my stay in Finland and allowing me to be part of his excellent group at FIMM. 

Thank you to all my NCRR colleagues, both new and old, that have made working at NCRR an absolute joy. A particular thank you to Clara Albiñana for starting her PhD journey alongside mine. It has been an incredible experience to have someone to go on this journey with. Finally, I would also like to thank my family and friends for being supportive and encouraging me during the entire PhD.



\mbox{}\\
\mbox{}\\
Emil Michael Pedersen\\
Aarhus, December 2022