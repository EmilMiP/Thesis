The aim of the dissertation is to present an approach to account for both family history and time in GWAS, as well as improve the predictive value of family history. This was achieved by estimating a liability with a modified liability threshold model that depends on information such as age of onset and family history. The thresholds used in the modified model are based on population representative cumulative incidence proportions stratified by sex and birth year. The following papers highlight different applications of the model.


\subsubsection{Paper 1: LT-FH++}
The first paper is the flagship paper of the dissertation. During the development of this paper, most of the implementation work was done, such that estimating the desired liability was possible. The work resulted in the method titled LT-FH++, which is an extension of the previously published method LT-FH. In short, LT-FH++ allows one to estimate a liability for an individual based on information such as age or age of onset, sex, birth year, and family history. This additional information can also be accounted for in each of the family members included, which was not possible with LT-FH. We found that the additional information did improve power, however in some cases it is only a modest improvement, since most of the power gain is driven by family history.

\subsubsection{Paper 2: ADuLT}
The second paper focused on the model underlying LT-FH++, called the age-dependent liability threshold (ADuLT) model, and its ability 
to increase power in GWAS compared to the more common Cox proportional hazards model. In this setting, the estimated liability depends 
on the same information as in the first paper, except we did not include family history and focused only on the age of onset aspect of 
the model. We only observed a notable difference between ADuLT and the CoxPH model when case ascertainment was present, but in such a 
case, the CoxPH was disproportionally affected and had a significantly lower power than ADuLT and even simple linear regression.

\subsubsection{Paper 3: fGRS}
