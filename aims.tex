The aims of the dissertation is to present an approach to account for both family history and time, as well as to improve the predictive value of family history in GWAS without increasing sample sizes. This was achieved by estimating a liability with a modified liability threshold model that depends on age-of-onset and family history. The thresholds used in the modified model are based on population representative cumulative incidence proportions stratified by sex and birth year. The following papers highlight different applications of the model.

\subsubsection{Paper 1: LT-FH++}
The first paper is the flagship paper of the dissertation. During the development of this paper, most of the implementation work was done such that estimating the desired liability was possible. The work resulted in the method titled LT-FH++, which is an extension of the previously published method LT-FH. LT-FH++ allows one to estimate a liability for an individual based on information such as age or age-of-onset, sex, birth year, and family history. This additional information can also be accounted for in each of the family members included, which was not possible with LT-FH. We found that the additional information did improve power, however in some cases it is only a modest improvement, since most of the power gain is driven by family history.

\subsubsection{Paper 2: ADuLT}
The second paper focused on the model underlying LT-FH++, called the age-dependent liability threshold (ADuLT) model, and its ability to increase power in GWAS compared to the more common Cox proportional hazards model. In this setting, the estimated liability depends on the same information as in the first paper, except we did not include family history and focused only on the age-of-onset aspect of the model. We only observed a notable difference between ADuLT and the Cox PH model when case ascertainment was present, but in such a case, the Cox PH was disproportionally affected and had a significantly lower power than ADuLT and even simple case-control linear regression.


\subsubsection{Paper 3: Family Liability}
The third paper focused on the predictive value of family history in the LT-FH++ model compared to the PRS and the conventional family history indicator. We estimate the liability in the LT-FH++ model, but excludes the index person's status and base the estimate solely on the family history. A multi-trait extension of the LT-FH++ model is also examined, where no disorder information is included on the index person. We found that the liability phenotype from LT-FH++ was significantly better than the binary family history variable, and the variance explained by the LT-FH++ phenotype was largely independent from the PRS. In the multi-trait case, the signal was still largely independent between the family history variables and the PRS, and the gap between The liability phenotype and the binary family history indicator was gone.
