The work presented in this dissertation focuses on the development, implementation, and application of what is now called LT-FH++. This model extends the previously published method called LT-FH, which is itself an extension of the classical LTM by Falconer\cite{falconer1965inheritance,falconer1967inheritance}. LT-FH extended the LTM to model family members for binary traits, allowing for the estimation of a genetic liability. When used as the outcome in a GWAS, the genetic liability lead to a significant power increase. LT-FH++ expanded this method even further by accounting for age-of-onset (in cases) or age (in controls), sex, and birth year in each considered individual. They are included in the model through population representative cumulative incidence proportions that are stratified by sex and birth year. By doing this, the threshold considered in the liability threshold model becomes unique for each individual. Individual thresholds have not previously been considered. Due to the fact that all individuals have a unique threshold, it was necessary to implement a computationally more efficient sampling strategy than the one used in LT-FH. As a result, LT-FH++ utilises a Gibbs sampler that efficiently samples from a truncated multivariate normal distribution with arbitrary thresholds. Furthermore, the Gibbs sampler was implemented in a way that enables computations to be performed in parallel, allowing for a better utilization of modern CPUs with many cores and high performance computing clusters.

In the first paper, \enquote{Accounting for age-of-onset and family history improves power in genome-wide association studies}\cite{pedersen2022accounting}, we considered most of the methodological development and implementation of LT-FH++. We showed that LT-FH++ performs between $ 4\% $ and $ 18\% $ better than LT-FH in terms of identifying the true causal associations in a simulated GWAS setup. Both LT-FH and LT-FH++ outperform the conventional case-control status and the GWAX phenotype. We assessed the performance of LT-FH++ in a non-simulated setup, by analysing mortality in the UK Biobank and four psychiatric disorders from iPSYCH. For mortality, LT-FH++ provided a large boost in power compared to LT-FH and case-control status. More precisely, we saw that LT-FH++ was able to identify $ 10 $ genome-wide significant SNPs, while LT-FH identified $ 2 $ and case-control status did not identify any. Across the four psychiatric disorders in iPSYCH, the difference between the three phenotypes was modest. There are likely several reasons for the lack of power gain over case-control status by LT-FH and LT-FH++, such as low family history prevalence and more polygenic disorders. Additional simulation studies also revealed that the mortality setup in UKBB was a near-perfect scenario for LT-FH++, as it benefits from a high prevalence in either the genotyped individuals or in the family history.

The second paper, \enquote{ADuLT: An efficient and robust time-to-event GWAS}, examined the best way to incorporate age-of-onset in a GWAS. More precisely, the paper compared ADuLT, the model underlying LT-FH++ that does not account for family history, to other time-to-event GWAS methods. The simplest and most commonly used time-to-event GWAS method is the Cox proportional hazards model. Since the Cox PH models are computationally intensive, most implementations are unable to handle more than $ 100,000 $ individuals. We will therefore use the most computationally efficient implementation called SPACox to represent these models. We compare the performance of ADuLT to that of SPACox and case-control status in a linear regression. We simulated genotypes and assigned phenotypes and age-of-onset under both the PH model and the LTM. As expected, the LTM-based methods performed better when phenotypes were assigned with this model, while SPACox performed best when the PH model was used to assign phenotypes. However, when we introduced case ascertainment, which means that the population was downsampled in order to have an equal number of cases and controls, a disproportional loss in power was observed in connection to SPACox. Conventionally, IPW is used to account for case ascertainment. Surprisingly, the IPW did not seem to have an effect, and SPACox still performed worse than simple linear regression, even under the PH model. The same disproportionate loss of power was also observed in the analyses of the iPSYCH disorders. As a result, we do not recommend using the PH model to identify genome-wide significant SNPs. A simple linear regression or ADuLT seem to be significantly better at identifying genome-wide significant SNPs. 

The third and final paper is not complete, but it has the working title, \enquote{Improving the predictive value of family history for psychiatric disorders},  and examines the predictive value of family history. Traditionally, family history is defined as a binary variable, that indicates whether the target individual has at least one family member, e.g.\ first degree, with the phenotype of interest. The predictive value of the binary family history indicator was compared to the PRS, as well as the LT-FH++ phenotype, where no information on the target individual was used. This means the LT-FH++ estimates the genetic liability solely based on the family history. The predictive value was assessed with the partial $ R^2 $ in a linear regression model. In order to compute the partial correlation, we defined a base model including age, sex, batch, and the first $ 20 $ principal components as covariates. The base model was then extended with either of the family history phenotypes, the PRS, or any combination of these three covariates, resulting in eight different models. Averaging across the $ 8 $ considered disorders, LT-FH++ had an increased partial $ R^2 $ of $ 19.6\% $ compared to the model including only the PRS. The binary family history variable had a $ 65.2\% $ \textit{decrease} compared to the PRS model. The model with both the binary family history variable and the LT-FH++ phenotype had almost the same predictive value as the model including only the LT-FH++ phenotype. Interestingly, the model with the LT-FH++ phenotype and the PRS had an almost additive increase in their predictive value, indicating that they capture independent genetic signals. The model with all three predictors performed nearly identically to the model with just the LT-FH++ phenotype and the PRS, indicating that the binary family history indicator does not provide much additional information. 
 
As most psychiatric disorders have a high genetic correlation, we also considered a regression model that included correlated phenotypes. The average prediction accuracy for of these phenotypes were higher than their single trait equivalent, but difference between LT-FH++ and the binary family history variable had largely disappeared, making both variables equally predictive. The model including both the LT-FH++ and the binary family history phenotype had nearly the same predictive power as a model accounting for either of them, meaning no new information was gained by using both. Combining the multi trait PRS with either of the family history phenotypes resulted in a model that had a predictive value close to the sum of the PRSs and the family history model's predictive values. As before, this indicates that the genetic signal captured by the PRSs and the family history phenotypes appear to be almost independent, even when accounting for the correlation between phenotypes.

In summary, we have successfully developed, implemented, and applied the LT-FH++ method in data sets and different ways. The LT-FH++ method provided improvements in each of the three applications that were considered, while remaining computationally efficient. Even though high quality information on family history and age-of-onset is not commonplace in all biobanks yet, we have demonstrated that the inclusion of such information can lead to better predictions and increase power in GWAS.

