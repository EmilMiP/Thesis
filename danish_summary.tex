Denne afhandling fokuserer på at udnytte familie historik og age-of-onset til at estimere en persons tilbøjelighed for en sygdom og til at forbedre statistisk styrke i GWAS. Dette opnås ved at udvikle, implementere og anvende en metode kaldet LT-FH++, som udvider den allerede udgivet LT-FH metode, som er baseret på den klassiske liability threshold model. LT-FH estimerer en genetisk tilbøjelighed til at blive syg baseret på familie historik, og det er allerede vist, at den genetiske tilbøjelighed kan forøge den statistiske styrke i GWAS. LT-FH++ udvider denne model yderligere ved også at tage højde for age-of-onset hos de sygdomsramte og alderen på kontrollerne, samt køn og fødselsår for alle inkluderet personer. Der tages højde for denne ekstra information igennem en populations repræsentativ kumulativ incidensproportion, som er stratificeret på baggrund af køn og fødselsår. Det tillader at modellen også kan tage højde for ascertainment bias når sygdomstilbøjeligheden estimeres. I praksis vil det betyde, at hver person kommer til at have en unik tærskelværdi i LTM. LT-FH++ benytter en beregningsmæssig effektiv Gibbs sampler, som sampler fra en trunkeret multivariat normalfordeling. Implementeringen kan paralleliseres og er skalerbar til at drage nytte af de mange CPU kerner, som er almindelige på moderne CPU'er, eller high performance computing clusters.

Afhandlingen består af tre dele, hvor hver del svarer til en artikel. I den første artikel blev LT-FH++ implementeret og benchmarket som en GWAS metode igennem simuleringer og anvendelse i UK biboank og iPSYCH. Den anden artikel undersøger modellen, som LT-FH++ er bygget på (age-dependent liability threshold model), som et robust og beregningsmæssig effektivt alternativ til andre state-of-the-art overlevelsesanalyse GWAS. Her benytter vi os af simuleringer og anvendelse i iPSYCH. I den tredje og sidste artikel benytter vi LT-FH++ til at estimere familie historie tilbøjeligheder for psykiatriske sygdomme og prædiktere sygdommene med disse. Vi udvider også modellen, således den er i stand til at tage højde for sigdomme. Her sammenligner vi også med en konventionel binær familie historik og PRS.