
With an ever-increasing number of biobanks that are able to link electronic health records to genetic data, it is important to find the best ways to properly utilise this information. The purpose of this paper was to examine the best way to include the age of onset information in a GWAS setting. The gold standard for modelling time to event is a survival analysis model. However, the adoption of methods such as Cox PH have been limited for GWAS. One of the main limiting factors for such a model is the computational cost associated with the analysis. Recent advances have allowed for CoxPH models and frailty models to be used on UKBB-sized biobanks \cite{bi2020fast,dey2022efficient}. Both utilise a saddle point approximation \cite{daniels1954saddlepoint}, as it provides a computationally efficient way to calculate p-values. Previously, such models have only been compared to other Cox models or to logistic regression, and only under the Cox PH generative model, potentially disadvantaging the logistic regression. A comparison has also not been performed when there is case ascertainment present in the data, which means either more or fewer cases than the general population. As many biobanks have some form of case ascertainment, it is important to make such a comparison. 

%However, we found that the Cox PH model performed worse than a simple linear regression when there was case ascertainment present in the data.

%The implementation of the proportional hazards model proposed by Bi et al. is called SPACox and is available as an R package. The frailty model proposed by Dey et al. is called GATE and has been implemented in R and Rcpp, which is a R-wrapper around C++.

Since a Cox PH model and a LTM are fundamentally different, we did not want to unfairly favour one method over the other. Therefore, we performed simulations under both generative models, meaning genotypes were simulated in the same way, but the phenotype was assigned with different generative models. Analysis were then performed in each scenario and with and without case ascertainment. One would expect that the LTM based methods would perform the best under the LTM model, and vice versa, which is also what we experienced. Interestingly, we found that SPACox was disproportionately affected by case ascertainment, suffering far more than the LTM based methods. With case ascertainment, SPACox had by far the lowest power under both generative models and all prevalences considered except for the least ascertained parameter setup under the proportional hazards model. Next, we applied all three methods to the iPSYCH data, which also has a high degree of case ascertainment. We performed a GWAS on ADHD, Autism, Depression, and Schizophrenia, as all of these phenotypes are ascertained for cases. The real-world analysis was in agreement with the simulations. 

Conventionally, inverse probability weighing (IPW) would be used to account for any form of ascertainment. As SPACox does not support IPW, we used the \texttt{coxph} function from the \textit{survival} R package \cite{Therneau2020-xf} for a GWAS with IPW in the simulations. We also considered a slightly smaller data set with all causal SNPs, but only $ 1000 $ null SNPs for comparison. However, it did not restore power to be on par with the LTM models. In fact, IPW did not seem to change the power in any noticeable way compared to SPACox. The test statistics used with IPW in the \texttt{coxph} function is based on a Wald test\cite{survivalVignette}, which means the test statistic is the estimate divided by the standard error. When performing IPW, the estimate will remain unbiased, but estimating the standard error can be difficult \cite{austin2016variance}. 


In summary, we found that when no case ascertainment was present ADuLT had the highest power under the LTM and was only slightly behind SPACox under the Cox PH generative model. With case ascertainment, we found that SPACox was disproportionally affected by case ascertainment, resulting in a significantly lower power to detect causal SNPs. We observed the same loss of power in real-world analysis in iPSYCH. This leads us to conclude that using Cox PH models as they are currently implemented to identify genome-wide significant SNPs is not recommended. Researchers should instead use other methods to identify the SNPs, such as linear regression or the ADuLT phenotype with your GWAS software of choice. The Cox PH models can then by used on a set of pre-identified SNPs for subsequent analysis. Another benefit of ADuLT is the opportunity to use family history information as well, which have already been shown to significantly increase power by several methods. Cox PH models do not have a way to include this information in a straight forward way, further limiting its power in comparison to alternatives.


%\textbf{THE VARIANCE ESTIMATE IS BASED ON HORVITZ-THOMPSEN}





%The cox ph test is a wald test, i.e. effect size / stderr. The effect size is estimated unbiased by weighing samples. the std err is based on a horvitx-thompsen estimator (https://cran.r-project.org/web/packages/survival/vignettes/survival.pdf) p. 39 , chap 2.7. 
% $ \hat{V}ar(\hat{\tau}_\pi)=\sum\limits_{i=1}^v \left( \dfrac{1-\pi_i}{\pi^2_i} \right) y^2_i + \sum\limits_{i=1}^v \sum\limits_{j\neq i} \left( \dfrac{\pi_{ij}-\pi_i \pi_j}{\pi_i \pi_j}\right)\dfrac{1}{\pi_{ij}} y_i y_j $

% this link discusses the horvitz-thompsen estimator in some detail. there are several pages.
%https://www150.statcan.gc.ca/n1/pub/12-001-x/2013001/article/11831/section4-eng.htm


%Wolter, K. (2007). Introduction to Variance Estimation, Second Edition. New York: Springer.
%https://link.springer.com/content/pdf/10.1007/978-0-387-35099-8.pdf
% chapter 2 in this book has a section on the sampling variance