
Few places in the world have as detailed, curated, and complete register information linked to genetic data as iPSYCH does. Recently, there have been a trend where biobanks such as UK biobank, DeCODE, and FinnGen have started linking to registers or supplement their genetic data with questionnaires. As a result, we strongly believe that the information stored in this supplementary information can be leveraged to increase statistical power to identify causal SNPs in a GWAS setting. Family history has previously been used to generate risk scores \cite{kannel1990contribution,splansky2007third} or been included as a covariate in epidemiological analysis \cite{ejlskov2021prediction,schendel2022evaluating}, and as such, is a parameter many researchers are familiar with and know its potential. Similarly, an entire branch of statistics is focused on modelling time-to-event, which means many researchers are also familiar with age-of-onset and recognise its potential. Here, we proposed LT-FH++ as a way to combine family history and age-of-onset distributions with the ordinary case-control status to increase power, thereby combining two previously separated types of analysis.

Simulations show that LT-FH++ does increase statistical power in a GWAS setting over LT-FH and case-control status. The exact power increase provided by LT-FH++ over LT-FH depends on the situation the method is applied to and varies from roughly $ 4\% $ to $ 18\% $. Through supplemental simulations we found that one can expect the highest increase in power with LT-FH++ over LT-FH, when cases are ascertained in the sample or in the sample's family members. The supplemental simulations have also provided valuable insight into the power difference in the real-world data analysis of UKBB and iPSYCH.

The mortality GWAS in UKBB highlights a near perfect example of LT-FH++'s potential. Death is the only guarantee in life, unlike many disorders that can be quite rare. The UKBB participants were between $ 40 $ to $ 69 $ years old at recruitment. This means many of the participant's parents have already passed or are close to their life expectancy and that the participants themselves are getting close to it. Therefore, death is prevalent among the parents and has an ever-increasing prevalence among the participants. Death has a modest prevalence in the participants, but a high prevalence among the parents. In summary, death satisfy both of the criteria for best case scenario for LT-FH++ that we identified from the simulations. 

In iPSYCH, the conditions for both LT-FH and LT-FH++ are not nearly as favourable. The largest source of power increase provided by LT-FH and LT-FH++ are from the family history information. LT-FH++ further refines this information with the age-of-onset distributions. Due to psychiatric disorders such as ADHD not being present in ICD-8, it limits the opportunity to diagnose many of the parents of the iPSYCH participants. This is true even though the iPSYCH participants are much younger than the UKBB participants. The design of iPSYCH also means that most affected siblings have already been selected, sequenced, and are themselves present in the data\cite{pedersen2018ipsych2012}. In summary, the family history seem to be lower than expected, due to the family either being sampled for iPSYCH or being too old to be easily diagnosed. However, even if an affected sibling pair is present and filtering would exclude one sibling's genotypes, their status would still increase the liability of the remaining sibling, which would not be the case for case-control status.

The polygenicity of the analysed phenotypes are also likely to be different. Death can have numerous sources, such as cancer, heart diseases, or accidents. Accidents are not likely to have a genetic signal, while cancers, heart diseases, smoking, etc. are. Some cancers and heart disease have one or more prominent genetic signals \cite{koyama2020population,marioni2018gwas}. On the other hand, psychiatric disorders have proven to be very polygenic, meaning there are many SNPs with a small effect size\cite{gandal2018shared}. This coupled with the relatively smaller sample size of iPSYCH compared to UKBB, may mean identifying genome-wide significant associations are harder, as smaller effect sizes generally require larger sample sizes.

Both LT-FH and LT-FH++ require additional information to estimate the underlying genetic liabilities. The availability of family history is still limited in practice for most biobanks, which limits their applicability. Unfortunately, the family history information cannot be acquired by means other than registers, questionnaires, etc. The same is not necessarily true for the CIPs. Within a biobank, information such as birth year, age-of-onset, and sex are often available to some extent. For instance, the age-of-onset may be slightly anonymised, such that the exact day or month may not be available, but a reasonable approximation is still known. The CIPs used by LT-FH++ are population representative and summarise the age-specific proportion of the considered phenotype. This means they can be used in different populations, as long as the populations and diagnosis are similar. As an example, CIPs derived from the Danish registers could be used with, e.g.\ other Scandinavian countries or the UK. As there are differences in diagnostic practices across countries, some care should be taken when using CIPs for other populations. For instance, if the CIPs are based on psychiatrists and the disorder of interest in a biobank is self reported. When using the CIPs in a different population, we would not recommend fixing the thresholds for cases, but rather let the lower limit be determined by the CIP and the upper limit be infinite. 