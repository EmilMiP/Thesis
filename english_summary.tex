This dissertation focuses on leveraging age-of-onset information and family history to better estimate disease liability and improve statistical power in genome-wide association studies. This is achieved through the  development, implementation, and application of a method called LT-FH++, which extends the previously published LT-FH method, and is based on the classical liability threshold model. LT-FH seeks to estimate a genetic liability based on family history, which has been shown to significantly increase power in GWAS. LT-FH++ extends this further by also accounting for age-of-onset in cases and age in controls, as well as sex, and birth year for all included individuals. This information is accounted for through population representative cumulative incidence proportions that are stratified by sex and birth year. This also allows the model to account for ascertainment biases when estimating disease liabilities. In practice, this leads to a threshold in the LTM that is unique to each individual. LT-FH++ utilises a computationally efficient Gibbs sampler that samples from a truncated multivariate normal distribution. The implementation is parallelizable and highly scalable for modern CPUs with many cores or high performance computing clusters. 

The thesis is in three parts, where each corresponds to one paper. The first part implements LT-FH++ and benchmarks it as a method for GWAS using both extensive simulations as well as UK biobank data and iPSYCH data. The second part examines the age-dependent liability threshold model (underlying LT-FH++) as a robust and computationally efficient survival analysis GWAS method, and benchmarks it against state-of-the art approaches using simulations and the iPSYCH data. The third last part focuses on using LT-FH++ for prediction based on family history liabilities for psychiatric disorders and extends the model to allow for correlated phenotypes. 
