During the dissertation, we have illustrated that the LT-FH++ (or ADuLT) phenotype has increased power in GWAS, outperformed standard survival GWAS methods when case ascertainment is present, and improved the predictive value of family history. LT-FH++ has managed to provide a computationally efficient link between the liability threshold model and survival models that can account for concepts such as censoring and family history. To the best of our knowledge, this link is a novel one that has not been examined or developed much yet. As a result, the first potential direction for future research is to examine this connection in greater detail, such that it will be possible to better understand how LT-FH++ fits in the existing survival analysis literature.

Conceptually, the genetic liability that LT-FH++ estimates share a lot of similarities with the purpose of the PRS. This relationship ought to be examined more, especially since the results from the third project of the dissertation showed an almost independent contribution from the LT-FH++ phenotype and the PRS. If they are conceptually the same, one would expect them to capture the same underlying signal, which did not seem to be the case in that project. Further examination of this relationship is therefore of particular interest. In a similar vein, if the PRS and LT-FH++ phenotype attempt to estimate the same underlying value, perhaps the PRS can be incorporated into the LT-FH++ model such that an even more accurate liability can be estimated. It would also be of interest to model the environmental covariance. It could have several benefits, such as improving the genetic liability estimate. Decomposing the environmental liability into risk factor driven liabilities is also of particular interest.

Furthermore, applying LT-FH++ to new data sets is also of interest. The stay abroad during the PhD was focused on applying LT-FH++ to the FinnGen data. Unfortunately, the project was not completed during the stay and due to time constraints in the PhD, has not been completed yet. However, there are currently plans to continue this project in the future. In the Danish registers, a multi generational register is also under development, which aims to create complete family trees from $ 1920 $ and onwards, which would allow for far larger family trees than what is currently possible. LT-FH++ has already been extended to allow for more than just parents and siblings, however a larger family tree may need to be considered. The multi generational register is an obvious area of application of LT-FH++. 


