
Over the couple of last decades, identifying genetic variants associated with diseases have been a major focus of research in human genetics; and for good reason. Identifying disease-associated single nucleotide polymorphisms (SNPs) or genes provides insight into the genetic architecture of diseases and their aetiology. Ultimately, improved understanding of the diseases can lead to novel treatments and development of preventive measures\cite{farkona2016cancer,goetz2018personalized}. Although the field of genomics is still relatively young several promising discoveries have already been made. Some notable achievements include the development of genetic screening methods for disorders through a polygenic risk score (PRS), identification of risk-associated genes to target for drug development\cite{goetz2018personalized}. In addition, those discoveries have shone light on the the aetiology of complex and polygenic disorders. Individual-level genotype data may further improve diagnoses and help identify more effective treatment options through precision medicine. A key method underlying these developments is the genome-wide association study (GWAS), which allows for the identification of SNPs and possible genes that are associated with a given phenotype. The detected SNPs can then be examined further with subsequent analysis and their risk contributions can be aggregated to construct a PRS\cite{prive2020ldpred2,lloyd2019improved,mak2017polygenic,yang2020accurate}. Therefore, it is important to continue to further improve GWAS methods and to increase the statistical power in GWAS settings.

Two primary methods that have been used to increase the statistical power of GWASs are to increase the sample size and to improve the methodology. As it is commonly not possible to share individual-level genotype data, the sample size has often been increased by meta-analysing GWASs from different cohorts. At the present time, several methods have been proposed to increase the amount of genetic heterogeneity captured between cohorts, such as inverse-variance weighted meta-analysis and random effect meta-analysis\cite{han2011random,willer2010metal}. The largest meta-analysis GWAS performed to date is of height with more than $ 5.4 $ million individuals\cite{yengo2022saturated}. Alongside the increased sample size, there have also been improvements regarding the methodology. These improvements have mainly been in connection to computational efficiency as well as the development of more powerful GWAS models. 

As the field evolves, knowledge about various complexities of GWAS are being discovered. This involves concepts such as in-sample relatedness (also called cryptic relatedness), different genetic ancestries, and population stratification \cite{genome2014whole,zeng2015statistical} . Initially, linear regression models were used to find associations in GWAS, but as these models are poorly suited to account for cryptic relatedness, differences in genetic ancestries, and population stratification, new models such as linear mixed models were proposed. BOLT-LMM\cite{loh2015efficient} is an excellent example of an improvement that provided both computational efficiency and a more complex model. Prior to its publication, linear mixed models had a prohibitive computational cost, making them intractable for analysis of more than $ 100,000 $ individuals.

Even though it is likely that further improvements regarding sample sizes and methodology will be made, it is also reasonable to consider related fields and their common practices, as the application of methods from other fields in human genetics has already lead to considerable improvements. Animal genetics has many similarities with human genetics and some of the commonly used models and computational strategies from the field of animal genetics have already been applied to human genetics with great success. As an example, some of the computational mechanisms employed by BOLT-LMM are based on methods commonly used in animal genetics, while the polygenic risk scores are heavily inspired by the genetic breeding value used in the same field.\cite{loh2015efficient,wray2019complex,meuwissen2001prediction}. Prior to the increase in size of genotyped (and imputed) data, family history has been a commonly used and valuable predictor for many disorders in both epidemiology as well as animal and human genetics\cite{guttmacher2004family,runeson2003family,collaborative2001familial,johns2001systematic}. A well known application of family history as a predictor in human genetics can be seen in the Framingham, where it was used to improve risk assessment of heart disease\cite{kannel1990contribution,splansky2007third}.

Unfortunately, family history is not commonly available with genetic data in biobanks. This has limited the development of methods that can utilise family history in a GWAS setting. There is a small, but fortunately an increasing, number of biobanks that provide some degree of family history with their genetic data. Among those biobanks are UK biobank (UKBB) \cite{bycroft2018uk}, deCODE\cite{noauthor_2012-jh}, iPSYCH\cite{bybjerg2020ipsych2015}, and FinnGen\cite{Kurki2022-pt}. Even if the family history is available, there are big differences in coverage and origin of the information between biobanks. In UKBB, family history is only available for $ 12 $ disorders and it was obtained through questionnaires. It is therefore likely be prone to recall-bias. The iPSYCH sample has been linked to the Danish registers, which allows for the construction of near complete family trees from $ 1969 $ onwards. Genetic and phenotypic information is available for all individuals in the iPSYCH sample, while all recorded family members have phenotypic information. As the name indicates, FinnGen originates from Finland, which (like Denmark) is known for its detailed registers. However, FinnGen has only limited family history linked to the genetic data due to privacy concerns. At the present time, it has only been allowed to link the parental cause of death to the genetic data stored in FinnGen, even though far more information would be available in the Finish registers. Even though the adoption of family history by biobanks has been limited, the family history methods that have been developed so far have shown a tremendous amount of potential. 

One of the first and most well-known family history methods that was developed is called genome-wide association study by proxy (GWAX)\cite{gwax}. GWAX redefines the binary case-control phenotype such that cases also include controls with family history. Cases under the GWAX approach are therefore either affected themselves or have close family members that are. Liu et al., who proposed GWAX, analysed Alzheimer's disease, which is a disorder with a low prevalence among the UKBB participants. Many of the participants are simply too young to have been diagnosed with Alzheimer's disease at the time of censoring. However, their parents are old enough to have been diagnosed prior to censoring, and hence, GWAX increased the number of considered cases. For low prevalence or late onset disorders, it has been shown to be a powerful tool when trying to identify genome-wide significant SNPs\cite{hujoel2020liability,gwax,pedersen2022accounting}. As a result, GWAX has successfully provided a proof-of-concept and paved the way for other family history methods. However, GWAX also has a limitation, since it loses power if the in-sample prevalence is high ($ >50\% $) for the GWAX phenotype \cite{gwax}. In addition, it is a heuristic method, which is not based on any model. Since GWAX was proposed, another method called the liability threshold model conditional on family history (LT-FH) has been introduced \cite{hujoel2020liability}. This method solves the two above mentioned limitations of GWAX. LT-FH is also the method that this dissertation have expanded further on to also allow for modelling of age-of-onset, sex, and cohort effects in the individual of interest and their considered family members. The extension we developed is called LT-FH++\cite{pedersen2022accounting}.

To the best of our knowledge, no other model accounts for family history \textit{and} age-of-onset simultaneously. In terms of age-of-onset, an often favoured method is some variation of the Cox proportional hazards(PH) models. In fact, the Cox PH model and the frailty model are the only two survival models that have been regularly used in GWAS settings\cite{bi2020fast,dey2022efficient}. The frailty model is a generalisation of the Cox PH model that also includes a random effect that can be used to model the cryptic relatedness. Frailty models and mixed models share many advantages, as they are both able to account for cryptic relatedness in biobanks. However, the adoption of frailty models in connection to GWASs has been limited. One of the reasons for the slow adoption is likely due the computational complexity of the frailty models. Prior to the publication of the Cox PH model called SPACox in $ 2020 $ by Bi et al., a Cox PH based GWAS was limited to less than $ 100.000 $ individuals due to computational cost\cite{bi2020fast}. This is especially striking, as other computationally intensive models, such as linear mixed models, had been computationally feasible for more than $ 400.000 $ individuals since $ 2015 $ \cite{loh2015efficient}. Frailty models were similarly computationally intractable for more than $ 20.000 $ individuals until $ 2022 $, where the method GATE was proposed\cite{dey2022efficient}. Both SPACox and GATE utilise the saddle point approximation(SPA) as an efficient way of calculating p-values. SPA only requires the cumulant generating function of the test statistic to calculate p-values.

In this dissertation, we will focus on the development and applications of LT-FH++, which is based on the age-dependent liability threshold model (ADuLT) \cite{pedersen2022adult}. If family history is included, we will refer to the method as LT-FH++, and if only the index person is considered (that is, no family history), we will refer to it as ADuLT. LT-FH++ combines many of the concepts used in survival analysis and the Cox PH methods that have been developed for GWASs with family history. In short, LT-FH extends the classical liability threshold model(LTM) proposed by Falconer\cite{falconer1965inheritance,falconer1967inheritance} to incorporate family history, while LT-FH++ extends LT-FH to further include age-of-onset information. LT-FH++ accounts for age-of-onset by using a personalised threshold in the LTM, instead of the fixed threshold that is normally used in the LTM. Each threshold used to determine the case-control status is redefined to depend on the age (for controls) or age-of-onset (for cases), birth year, and sex. LT-FH++ incorporates family history and a population representative cumulative incidence proportions (CIPs), which makes it possible to account for censoring and stratification by sex and birth year in a liability threshold setup. Details on LT-FH and LT-FH++ are given in \cref{sec:methods:LTMs}.

It is important to highlight that the family history methods have a clear advantage in that they can be used as a replacement for any phenotype in any further analysis. Going back to the example about Alzheimer's disease, then the GWAX phenotype does not require any changes to the analysis plan. The only change is that the case-control status has been replaced by the GWAX phenotype. The same holds for the LT-FH phenotype, but with LT-FH there is no worry of any potential power loss, as it will always outperform GWAX and case-control phenotypes \cite{hujoel2020liability,pedersen2022accounting}. This is also true for LT-FH++ over LT-FH (and by extension GWAX). Since all of these refined family history phenotypes can be used as replacements for the original phenotypes, the methodological benefits can be used immediately and do not require further implementation or modification to make them compatible with the chosen GWAS software. For example, GWAS with a family history phenotype as the outcome can be performed with linear regression, or swapped to a linear mixed model with no other change. This means the family history methods builds on top of the methodological improvements that happen in parallel.

It is important to highlight that this is different from the survival GWAS methods that have been proposed, as they are all model specific implementations\cite{dey2022efficient,bi2020fast,Therneau2020-xf}. Each new model is not compatible with previous implementations. An example of this is SPACox and GATE. Both SPACox and GATE suffer from this drawback, as they invalidate any previously implemented Cox PH or frailty method in a GWAS setting due to their applicability to large datasets. It also implies that any possible future model accounting for an aspect that available models do not account for at the present time, most likely will invalidate both SPACox and GATE, while LT-FH++ can be combined with future implementations immediately. LT-FH++ is therefore in a strong position, as it enables the user to immediately utilise new methodological advancements, while preserving the survival analysis aspect and its inherent power increase.    

Many human traits are highly polygenic, which makes it difficult to identify the underlying mechanisms that causes the traits\cite{song2021selection}. In particular, many psychiatric disorders have poorly understood biological mechanisms, are highly polygenic, and are very heritable\cite{pardinas2018common,esteller2020genomic}. Because of this, polygenic traits such as psychiatric disorders often require larger sample sizes or better models compared to less polygenic traits\cite{han2008genome,bergen2012genome,badano2002beyond}. Utilising additional information to assist in increasing statistical power for such phenotypes are therefore of particular interest, as it does not require additional individuals to be sequenced to increase power.



