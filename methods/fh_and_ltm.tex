
This section is deal with how to account for family history in a GWAS setting on a phenotypic level rather than a genetic one, where the main focus has been attempting to eliminate any potential problems with including related samples. The research into accounting for family history on a phenotypic level has been very limited. This is likely due to the relatively low occurrence of family history variables in conjunction with genotype data. There have been some biobanks, such as UK biobank, DeCODE, iPSYCH, and FinnGen, where some level of family history information have been linked with genotypes.

% The main methods that account for FH is GWAX, LT-FH, and our method, LT-FH++. GWAX is not a model based approach, but rather a heuristic way to account for family history. LT-FH and LT-FH++ are both model based approaches that can account for family history. They build on the same ideas of extending the liability threshold model...[now make a section on the model, math, and how we estimate the (genetic) liability of an individual]
% GWAX

% LT-FH

% 






