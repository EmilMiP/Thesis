All projects in this dissertation are based on two types of information. The first is register data and the second is genotype data. The registers are used to define the study population, acquire phenotype information for individuals, and link family members. The genotype data served as the basis for a genome-wide association study that the dissertation is trying to increase power for without increasing the sample size. 

\subsection{Danish registers}
The Danish registers serve as the main source of phenotypic information and allows us to link individuals to their family members. The registers can be linked to one another through a unique 10 digit number assigned to every Dane and resident in Denmark since 1968.  

\subsubsection{The civil registration system}
The Danish civil registration system was established on 2 April 1968, and all persons living in Denmark were registered for administrative use. All registered individuals were given a 10 digit unique personal identification number, commonly referred to as the CPR-number. The CPR-number is used to link individuals across all registers. The register holds information on name, gender, date of birth, place of birth, citizenship, identity of parents, and it will be continually updated with information on vital status, place of residence and spouses. On 1 May 1972 all persons living in Greenland were also included into the register. \cite{pedersen2011danish}

\subsubsection{The national patient register}
The Danish national patient register was established in 1977. Its contents has been expanded several times since it was created. Originally, it contained only information on patients admitted to somatic wards. In 1995, the register was expanded to also include outpatients, patients from emergency rooms, and patients from psychiatric wards. In 1994, the international classification of disease, version 10, was adopted in Denmark, and prior to the adoption, version 8 was used. \cite{lynge2011danish}


\subsubsection{The psychiatric central research register}
The psychiatric central research register has valid data from 1970 and onwards. At the beginning, the register contained information on every admission to a mental hospital and psychiatric department, where information such as dates of onset, end of treatment, and all diagnosis were recorded. In 1995, the register became an integrated part of the Danish national patient register and was expanded to also record information from psychiatric emergency room and outpatient treatment. Similarly, ICD-10 codes were used after 1995, and ICD-8 were used before. Most mild and moderate disorders are treated by general practitioners or in private practices and will therefore not be in the register.\cite{mors2011danish}



\subsection{Genotype data}


\subsubsection{iPSYCH}

\subsubsection{UK biobank}


